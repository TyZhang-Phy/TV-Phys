\documentclass[12pt]{article}
\usepackage{ctex}
\usepackage{graphicx}
\usepackage{subfigure}
\usepackage{caption}
\usepackage{float}
\usepackage{physics}
\usepackage{amsmath}
\usepackage{geometry}
\geometry{left=2.5cm,right=2.5cm,top=2.5cm,bottom=2.5cm}
\title{Homework 01}
\author{Zhang Tingyu $\ $ 35206402}
\graphicspath{{figures/}}

\begin{document}

\maketitle

\section{}

The nuclear reaction in the experiment where Rutherford observed protons is
\begin{equation*}
    \ ^{14}N+\alpha\rightarrow\ ^{17}O+p.
\end{equation*}
According to the mass of proton and each atom, the energy of $\alpha$ should above
\begin{equation*}
    M(^{17}O)+M(p)-M(^{14}N)-M(\alpha)=1.279\times10^{-3}{\rm amu}.
\end{equation*}
Now, consider this nuclear reaction
\begin{equation*}
    \ ^{16}O+\alpha\rightarrow\ ^{19}F+p.
\end{equation*}
The energy of $\alpha$ in this case should above
\begin{equation*}
    M(^{19}F)+M(p)-M(^{16}O)-M(\alpha)=8.711\times10^{-3}{\rm amu}
\end{equation*}
which is much higher than that in the case of $\rm N_2$ gas. This is why Rutherford 
observed protons with $\rm N_2$ gas instead of $\rm O_2$.
To observe proton with $\rm O_2$ gas, the energy needed of $\alpha$ is 
\begin{equation*}
    8.711\times10^{-3}{\rm amu}\approx8.134{\rm MeV}
\end{equation*}

\section{}

The rest mass of proton is $M_0(p)$
\begin{equation*}
    E_0(p)=M_0(p)c^2=9.4085\times10^2{\rm MeV}
\end{equation*}
The Total energy of the proton is 
\begin{equation*}
    E(p)=\nu M_0(p)c^2=E_0(p)+5.7{\rm MeV}=9.4655\times10^2{\rm MeV}
\end{equation*}
where
\begin{equation*}
    \nu=\frac{1}{\sqrt{1-(v/c)^2}}=1.00606
\end{equation*}
\begin{equation*}
    \Rightarrow v=0.32878\times10^8{\rm m/s}
\end{equation*}
where $v$ is the velocity of the outgoing proton. And the momentum of the proton 
is 
\begin{equation*}
    P(p)=\nu M_0(p)v
\end{equation*} 
We suppose that the deflection angle of the $\gamma$-ray is $\pi$. 
In this case, we obtain the minimum energy of the $\gamma$-ray:
\begin{equation*}
    E(\gamma)=\frac{P(p)}{2}c=8.29885\times10^{-12}{\rm J}\approx51.868{\rm MeV}
\end{equation*} 
according to the conservation of momentum.

\end{document}